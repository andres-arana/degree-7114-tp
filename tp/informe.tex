\documentclass[a4paper,11pt]{article}

%%%%%%%%%%%%%%%%%%%%%%%%%%%%%%%%%%%%%%%%%%%%%%%%%%%%%%%%%%%%%%%%%%%%%%%%
% Paquetes utilizados
%%%%%%%%%%%%%%%%%%%%%%%%%%%%%%%%%%%%%%%%%%%%%%%%%%%%%%%%%%%%%%%%%%%%%%%%

% Gráficos complejos
\usepackage{graphicx}
\usepackage{caption}
\usepackage{subcaption}
\usepackage{placeins}

% Soporte para el lenguaje español
\usepackage{textcomp}
\usepackage[utf8]{inputenc}
\usepackage[T1]{fontenc}
\DeclareUnicodeCharacter{B0}{\textdegree}
\usepackage[spanish]{babel}

% Matemáticos
\usepackage{amssymb,amsmath}

% Tablas complejas
\usepackage{multirow}

% Formato de párrafo
\setlength{\parskip}{1ex plus 0.5ex minus 0.2ex}

%%%%%%%%%%%%%%%%%%%%%%%%%%%%%%%%%%%%%%%%%%%%%%%%%%%%%%%%%%%%%%%%%%%%%%%%
% Documento
%%%%%%%%%%%%%%%%%%%%%%%%%%%%%%%%%%%%%%%%%%%%%%%%%%%%%%%%%%%%%%%%%%%%%%%%

\begin{document}

\tableofcontents
\clearpage

\part{Modelo propuesto por el grupo de trabajos prácticos}

\section{Objetivo}

Determinar la cantidad a producir de cada uno de los productos y las compras de
materia prima correspondientes en un mes para maximizar el beneficio neto
obtenido de las ventas de los productos menos los costos asociados a su
producción.

\section{Hipótesis}

\begin{enumerate}

  \item Todo lo producido es vendido. No existe beneficio económico alguno en
    acumular stock de los productos.

  \item Las proyecciones de demanda actual son exactas. Se puede vender hasta
    exactamente cada uno de los valores preestablecidos.

  \item No existe inflación o, en caso de existir, la relación entre los
    precios de costos y de ventas se mantiene constante en el periodo
    analizado.

  \item El proceso de mezclado de los productos produce una mezcla
    completamente homogénea. Cada producto final embolsado posee una
    composición de maíz, alfalfa, sorgo y cáscara de arroz en igual porcentaje
    a la composición correspondiente al total producido para el producto.

  \item Se dispone de materia prima y mano de obra suficiente para producir
    cualquier cantidad de producto. El único factor limitante son las horas
    mensuales del proceso de mezclado.

  \item No existen costos adicionales de mano de obra, o bien están incluidos
    en el costo del proceso de mezclado.

  \item No hay ningún tipo de merma en todo el proceso productivo. El peso
    total combinado de la materia prima comprada corresponde con el peso total
    de lo producido.

  \item Se supone que la cáscara de arroz no aporta ningún tipo de nutriente,
    por ser de difícil digestión. Es decir, tiene \(0\%\) carbohidratos,
    \(0\%\) proteínas y \(0\%\) fibras.

\end{enumerate}

\section{Modelo}

\subsection{Variables utilizadas}

\begin{table}[h!t]
  \centering
  \begin{tabular}{ | c | c | p{7cm} | }
    \hline
    \textbf{Variable} & \textbf{Unidades}          & \textbf{Descripción} \\ \hline
     \(P_p\)          & \(\text{kg} / \text{mes}\) & Cantidad producida mensualmente para el producto PF. \\ \hline
     \(P_c\)          & \(\text{kg} / \text{mes}\) & Cantidad producida mensualmente para el producto CF. \\ \hline
     \(P_g\)          & \(\text{kg} / \text{mes}\) & Cantidad producida mensualmente para el producto GF. \\ \hline
  \end{tabular}
  \caption{Variables relacionadas con las cantidades de productos}
\end{table}

\begin{table}[h!t]
  \centering
  \begin{tabular}{ | c | c | p{7cm} | }
    \hline
    \textbf{Variable} & \textbf{Unidades}          & \textbf{Descripción} \\ \hline
     \(C_m\)          & \(\text{kg} / \text{mes}\) & Cantidad total comprada mensualmente para el maíz. \\ \hline
     \(C_a\)          & \(\text{kg} / \text{mes}\) & Cantidad total comprada mensualmente para el alfalfa. \\ \hline
     \(C_s\)          & \(\text{kg} / \text{mes}\) & Cantidad total comprada mensualmente para el sorgo. \\ \hline
     \(C_c\)          & \(\text{kg} / \text{mes}\) & Cantidad total comprada mensualmente para la cáscara de arroz. \\ \hline
  \end{tabular}
  \caption{Variables relacionadas con las cantidades de las materias primas compradas.}
\end{table}

\begin{table}[h!t]
  \centering
  \begin{tabular}{ | c | c | p{7cm} | }
    \hline
    \textbf{Variable} & \textbf{Unidades}          & \textbf{Descripción} \\ \hline
              \(E_m\) & \(\text{kg} / \text{mes}\) & Cantidad comprada mensualmente en exceso sobre lo pactado para el maíz. \\ \hline
              \(D_m\) & \(\text{kg} / \text{mes}\) & Cantidad comprada mensualmente en defecto sobre lo pactado para el maíz. \\ \hline
              \(E_a\) & \(\text{kg} / \text{mes}\) & Cantidad comprada mensualmente en exceso sobre lo pactado para el alfalfa. \\ \hline
              \(D_a\) & \(\text{kg} / \text{mes}\) & Cantidad comprada mensualmente en defecto sobre lo pactado para el alfalfa. \\ \hline
              \(E_s\) & \(\text{kg} / \text{mes}\) & Cantidad comprada mensualmente en exceso sobre lo pactado para el sorgo. \\ \hline
              \(D_s\) & \(\text{kg} / \text{mes}\) & Cantidad comprada mensualmente en defecto sobre lo pactado para el sorgo. \\ \hline
              \(E_c\) & \(\text{kg} / \text{mes}\) & Cantidad comprada mensualmente en exceso sobre lo pactado para la cáscara de arroz. \\ \hline
              \(D_c\) & \(\text{kg} / \text{mes}\) & Cantidad comprada mensualmente en defecto sobre lo pactado para cáscara de arroz. \\ \hline
              \(E_h\) & \(\text{kg} / \text{mes}\) & Cantidad de producción total elaborada en exceso sobre el límite establecido para la producción en régimen regular. \\ \hline
              \(D_h\) & \(\text{kg} / \text{mes}\) & Cantidad de producción total elaborada en defecto sobre el límite establecido para la producción en régimen regular. \\ \hline
  \end{tabular}
  \caption{Variables relacionadas con el exceso por sobre las diferentes metas de producción o compra de materias primas.}
\end{table}

\begin{table}[h!t]
  \centering
  \begin{tabular}{ | c | c | p{7cm} | }
    \hline
    \textbf{Variable} & \textbf{Unidades}          & \textbf{Descripción} \\ \hline
           \(M_{mp}\) & \(\text{kg} / \text{mes}\) & Cantidad de maíz utilizado en la mezcla del producto PF. \\ \hline
           \(M_{mc}\) & \(\text{kg} / \text{mes}\) & Cantidad de maíz utilizado en la mezcla del producto CF. \\ \hline
           \(M_{mg}\) & \(\text{kg} / \text{mes}\) & Cantidad de maíz utilizado en la mezcla del producto GF. \\ \hline
           \(M_{ap}\) & \(\text{kg} / \text{mes}\) & Cantidad de alfalfa utilizado en la mezcla del producto PF. \\ \hline
           \(M_{ac}\) & \(\text{kg} / \text{mes}\) & Cantidad de alfalfa utilizado en la mezcla del producto CF. \\ \hline
           \(M_{ag}\) & \(\text{kg} / \text{mes}\) & Cantidad de alfalfa utilizado en la mezcla del producto GF. \\ \hline
           \(M_{sp}\) & \(\text{kg} / \text{mes}\) & Cantidad de sorgo utilizado en la mezcla del producto PF. \\ \hline
           \(M_{sc}\) & \(\text{kg} / \text{mes}\) & Cantidad de sorgo utilizado en la mezcla del producto CF. \\ \hline
           \(M_{sg}\) & \(\text{kg} / \text{mes}\) & Cantidad de sorgo utilizado en la mezcla del producto GF. \\ \hline
           \(M_{cp}\) & \(\text{kg} / \text{mes}\) & Cantidad de cáscara de arroz utilizado en la mezcla del producto PF. \\ \hline
           \(M_{cc}\) & \(\text{kg} / \text{mes}\) & Cantidad de cáscara de arroz utilizado en la mezcla del producto CF. \\ \hline
           \(M_{cg}\) & \(\text{kg} / \text{mes}\) & Cantidad de cáscara de arroz utilizado en la mezcla del producto GF. \\ \hline
  \end{tabular}
  \caption{Variables relacionadas con las mezclas de las materias primas constituyentes de los diferentes productos.}
\end{table}

\FloatBarrier

\subsection{Planteo matemático}

Se desea maximizar el siguiente funcional:

\[
  \text{MAX} Z = \text{V} - \text{CC} - \text{CM} - \text{CE}
\]

El funcional se compone de los ingresos por ventas V, los costos por compras de
materia prima CC, los costos por mezclado CM y los costos por embolsado CE.
Cada uno de los términos se encuentran detallados en las siguientes ecuaciones:

\begin{equation}\label{eq:ventas}
  \text{V} = 15 P_p + 12 P_c + 10 P_g \\
\end{equation}

\begin{equation}\label{eq:cc}
  \text{CC} = (5 C_m + 8 C_a + 10 C_s + 2 C_c) + (0.5 E_m + 0.8 E_a + 1 E_s + 0.2 E_c) \\
\end{equation}

\begin{equation}\label{eq:cm}
  \text{CM} = 2 (P_p + P_c + P_g) + 1 E_h \\
\end{equation}

\begin{equation}\label{eq:ce}
  \text{CE} = 0.5 (P_p + P_c + P_g)
\end{equation}

En la ecuación \ref{eq:ventas} se detallan los ingresos por los productos
vendidos de acuerdo a los precios establecidos.

En la ecuación \ref{eq:cc} se detallan los costos por las compras de materia
prima. Los mismos se componen de dos partes, por un lado el costo por el precio
pactado hasta el límite acordado y por el otro los recargos del \(10\%\) para
todo lo que sea exceso por sobre el máximo disponible de acuerdo al pacto.

En la ecuación \ref{eq:cm} se detallan los costos por el mezclado de las
materias primas al preparar los productos. Nuevamente se discriminan dos
componentes: por un lado el costo original por debajo del límite del régimen
regular y por el otro el recargo del \(50\%\) en régimen de tiempo extra.

En la ecuación \ref{eq:ce} se detallan los costos por el embolsado.

Cada una de las materias primas compradas se distribuye en las mezclas de los
productos elaborados:

\[
  C_m = M_{mp} + M_{mc} + M_{mg}
\]
\[
  C_a = M_{ap} + M_{ac} + M_{ag}
\]
\[
  C_s = M_{sp} + M_{sc} + M_{sg}
\]
\[
  C_c = M_{cp} + M_{cc} + M_{cg}
\]

Además, cada producto se constituye por la mezcla de las materias primas
involucradas:

\[
  P_p = M_{mp} + M_{ap} + M_{sp} + M_{cp}
\]
\[
  P_c = M_{mc} + M_{ac} + M_{sc} + M_{cc}
\]
\[
  P_g = M_{mg} + M_{ag} + M_{sg} + M_{cg}
\]

Cada producto tiene una composición particular de nutrientes que debe
respetarse. Para PF, la composición de nutrientes se representa como sigue:

\[
  0.3 M_{mp} + 0.25 M_{ap} + 0.25 M_{sp} \leq 0.4 P_p
\]
\[
  0.12 M_{mp} + 0.15 M_{ap} + 0.18 M_{sp} \geq 0.1 P_p
\]
\[
  0.58 M_{mp} + 0.60 M_{ap} + 0.57 M_{sp} \leq 0.6 P_p
\]

Para CF, la composición es la siguiente:

\[
  0.3 M_{mc} + 0.25 M_{ac} + 0.25 M_{sc} \leq 0.5 P_c
\]
\[
  0.12 M_{mc} + 0.15 M_{ac} + 0.18 M_{sc} \geq 0.12 P_c
\]

Para GF, la composición es:

\[
  0.3 M_{mg} + 0.25 M_{ag} + 0.25 M_{sg} \leq 0.35 P_g
\]
\[
  0.12 M_{mg} + 0.15 M_{ag} + 0.18 M_{sg} \geq 0.15 P_g
\]

Por otro lado, las estimaciones de demanda suponen un límite a la producción:

\[
  P_p \leq 5000
\]
\[
  P_c \leq 8000
\]
\[
  P_g \leq 2500
\]

Las cantidades de materia prima pactadas y los excesos correspondientes se
modelan como sigue:

\[
  C_m - 7000 = E_m - D_m
\]
\[
  C_a - 5000 = E_a - D_a
\]
\[
  C_s - 2000 = E_s - D_s
\]
\[
  C_c - 10000 = E_c - D_c
\]

Por último, el régimen de mezclado regular implica que pueden procesarse hasta
\(70 (\text{kg} / \text{h}) 200 (\text{hs} / \text{mes}) = 14.000 (\text{kg} /
\text{mes})\) en régimen regular, y hasta \(60 (\text{kg} / \text{h}) 50
(\text{hs} / \text{mes}) = 3.000 (\text{kg} / \text{mes})\) en régimen de
tiempo extra. Esto implica que las condiciones de excesos de horas en régimen
de tiempo extra y los límites de producción asociados al proceso de mezclado
se pueden representar por:

\[
  (P_p + P_c + P_g) - 14000 = E_h - D_h
\]
\[
  E_h \leq 3000
\]

\clearpage
\part{Modelo propuesto por el turno}

\section{Objetivo}

Determinar la cantidad de máquinas virtuales A y B a ejecutar en el servidor de
virtualización durante un periodo indeterminado para maximizar el ahorro
generado por las máquinas virtuales en ejecución.

\section{Hipótesis}

\begin{enumerate}

  \item No hay desperdicio de RAM o disco rígido por efectos de padding. Los
    recursos utilizados por las máquinas virtuales instaladas son equivalentes
    a la suma de los recursos ocupados individualmente por cada una de ellas.

  \item No hay consumo de ningún tipo de recurso por parte del software de
    virtualización en el servidor. Los recursos del mismo están completamente
    avocados a las máquinas virtuales individuales.

  \item No hay inflación, o si la hay afecta de la misma manera a los montos de
    ahorro individuales de cada máquina.

  \item No hay límite en la cantidad de máquinas virtuales que se pueden
    instalar. El único límite está en los recursos disponibles en el servidor
    de virtualización.

  \item No hay costos de ningún tipo por instalar copias adicionales de una
    máquina virtual.

\end{enumerate}

\section{Modelo}

\subsection{Variables utilizadas}

\begin{table}[h!t]
  \centering
  \begin{tabular}{ | c | c | p{7cm} | }
    \hline
    \textbf{Variable} & \textbf{Unidades} & \textbf{Descripción} \\ \hline
                    A & Instancias        & Cantidad de máquinas virtuales A instaladas. \\ \hline
                    B & Instancias        & Cantidad de máquinas virtuales B instaladas. \\ \hline
  \end{tabular}
  \caption{Variables utilizadas en la resolución del modelo.}
\end{table}

\FloatBarrier

\subsection{Planteo matemático}

Se desea maximizar el siguiente funcional:

\[
  \text{MAX} Z = 5000 A + 3000 B
\]

Los recursos disponibles en los servidores de virtualización condicionan la
cantidad de máquinas virtuales instalables de acuerdo a las siguientes
ecuaciones:

\begin{equation}
  2 A + 1 B \leq 15
\end{equation}
\begin{equation}
  6 A + 4 B \leq 60
\end{equation}
\begin{equation}
  1 A + 3 B \leq 40
\end{equation}

Por otro lado, la cantidad de máquinas virtuales mínimas a instalar se modela
como sigue:

\begin{equation}
  A + B \geq 12
\end{equation}

\end{document}
