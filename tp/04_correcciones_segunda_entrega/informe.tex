\documentclass[a4paper,11pt]{article}

%%%%%%%%%%%%%%%%%%%%%%%%%%%%%%%%%%%%%%%%%%%%%%%%%%%%%%%%%%%%%%%%%%%%%%%%
% Paquetes utilizados
%%%%%%%%%%%%%%%%%%%%%%%%%%%%%%%%%%%%%%%%%%%%%%%%%%%%%%%%%%%%%%%%%%%%%%%%

% Gráficos complejos
\usepackage{graphicx}
\usepackage{caption}
\usepackage{subcaption}
\usepackage{placeins}

% Soporte para el lenguaje español
\usepackage{textcomp}
\usepackage[utf8]{inputenc}
\usepackage[T1]{fontenc}
\DeclareUnicodeCharacter{B0}{\textdegree}
\usepackage[spanish]{babel}

% Matemáticos
\usepackage{amssymb,amsmath}

% Tablas complejas
\usepackage{multirow}

% Formato de párrafo
\setlength{\parskip}{1ex plus 0.5ex minus 0.2ex}

% Formato de encabezados y pies de página
\usepackage{fancyhdr}
\fancyhead{}
\fancyfoot{}
\setlength\headheight{40pt}
\fancyhead[L]{
  Facultad de Ingeniería \\ 
  Universidad de Buenos Aires
}
\fancyhead[R]{
  71.14 - Modelos y Optimización I \\
  Trabajo Práctico - Correcciones 2° entrega
}
\fancyfoot[C]{
  \thepage
}
\pagestyle{fancy}

%%%%%%%%%%%%%%%%%%%%%%%%%%%%%%%%%%%%%%%%%%%%%%%%%%%%%%%%%%%%%%%%%%%%%%%%
% Documento
%%%%%%%%%%%%%%%%%%%%%%%%%%%%%%%%%%%%%%%%%%%%%%%%%%%%%%%%%%%%%%%%%%%%%%%%
\title{\textbf{Trabajo Práctico} - Correcciones 2° entrega}
\author{Andrés Gastón Arana, \textit{P. 86.203}}
\date{}

\begin{document}

\maketitle
\clearpage

\part{Modelo propuesto por el grupo de trabajos prácticos}

\section{Resolución por Simplex}

En la primer entrega se encontró un error de cálculos en la tabla óptima de
simplex. A continuación en el cuadro \ref{tab:simplex} se presentan nuevamente
las tablas intermedias y la nueva tabla óptima:

\begin{table}[h!]
\centering
\begin{tabular}{ | c | c | c || c | c | c | c | c | c | c || c | }
  \hline
  C                               & \(X_i\)   & B         & \(A_1\)   & \(A_2\)     & \(A_3\)     & \(A_4\) & \(A_5\) & \(A_6\)          & \(A_m\) & \(\theta\) \\ \hline

  0 & \(x_3\) & 15 & \textbf{2} & 1 & 1 & 0 & 0 & 0  & 0 & \(7.5\) \\ \hline
  0 & \(x_4\) & 60 & 6          & 4 & 0 & 1 & 0 & 0  & 0 & 10 \\ \hline
  0 & \(x_5\) & 40 & 1          & 3 & 0 & 0 & 1 & 0  & 0 & 40 \\ \hline
 -M & \(m\)   & 12 & 1          & 1 & 0 & 0 & 0 & -1 & 1 & 12 \\ \hline \hline
 \multicolumn{3}{|c|}{ \(Z = 0\)}                         & -M - 5000 & -M - 3000 & 0       & 0           & 0           & M       & 0       & \\ \hline \hline

  5000 & \(x_1\) & 7,5  & 1 & 0,5          & 0,5  & 0 & 0 & 0  & 0 & 15 \\ \hline
     0 & \(x_4\) & 15   & 0 & 1            & -3   & 1 & 0 & 0  & 0 & 15 \\ \hline
     0 & \(x_5\) & 32,5 & 0 & 2,5          & -0,5 & 0 & 1 & 0  & 0 & 13 \\ \hline
    -M & \(m\)   & 4,5  & 0 & \textbf{0,5} & -0,5 & 0 & 0 & -1 & 1 & 9 \\ \hline \hline
  \multicolumn{3}{|c|}{ \(Z = 37500 - Mm\)}               & 0       & -500 - 0,5M & 2500 + 0,5M & 0       & 0       & M                & 0       & \\ \hline \hline

    5000   & \(x_1\) & 3  & 1 & 0 & 1  & 0 & 0 & 1          & -1 & 3 \\ \hline
       0   & \(x_4\) & 6  & 0 & 0 & -2 & 1 & 0 & 2          & -2 & 3 \\ \hline
        0  & \(x_5\) & 10 & 0 & 0 & 2  & 0 & 1 & \textbf{5} & -5 & 2 \\ \hline
      3000 & \(x_2\) & 9  & 0 & 1 & -1 & 0 & 0 & -2         & 2  & \\ \hline \hline
   \multicolumn{3}{|c|}{ \(Z = 42000\)}                   & 0       & 0           & 2000        & 0       & 0       & -1000            & 1000 + M & \\ \hline \hline

  5000                                                    & \(x_1\) & 1  & 1 & 0 & 0,6  & 0 & -0,2 & 0 & 0  & \\ \hline
                                                        0 & \(x_4\) & 2  & 0 & 0 & -2,8 & 1 & -0,4 & 0 & 0  & \\ \hline
                                                        0 & \(x_6\) & 2  & 0 & 0 & 0,4  & 0 & 0,2  & 1 & -1 & \\ \hline
                                                     3000 & \(x_2\) & 13 & 0 & 1 & -0,2 & 0 & 0,4  & 0 & 0  & \\ \hline \hline
                     \multicolumn{3}{|c|}{ \(Z = 44000\)} & 0 & 0 & 2400 & 0 & 200 & 0 & M & \\ \hline

\end{tabular}
\caption{Tablas simplex para el problema}\label{tab:simplex}
\end{table}

\FloatBarrier

\section{Análisis de los resultados}

Al análisis de los resultados de la primer entrega se le añaden los siguientes
puntos:

\begin{itemize}

  \item Hay sobrante de espacio en el disco del servidor, con 2 GB en exceso.

  \item Se cumplió la restricción de la cantidad mínima de máquinas virtuales
    (12) en exceso. Se instalarán 14 máquinas, con 2 máquinas en exceso.

\end{itemize}

\end{document}
