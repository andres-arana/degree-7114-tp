\documentclass[a4paper,11pt]{article}

%%%%%%%%%%%%%%%%%%%%%%%%%%%%%%%%%%%%%%%%%%%%%%%%%%%%%%%%%%%%%%%%%%%%%%%%
% Paquetes utilizados
%%%%%%%%%%%%%%%%%%%%%%%%%%%%%%%%%%%%%%%%%%%%%%%%%%%%%%%%%%%%%%%%%%%%%%%%

% Gráficos complejos
\usepackage{graphicx}
\usepackage{caption}
\usepackage{subcaption}
\usepackage{placeins}

% Soporte para el lenguaje español
\usepackage{textcomp}
\usepackage[utf8]{inputenc}
\usepackage[T1]{fontenc}
\DeclareUnicodeCharacter{B0}{\textdegree}
\usepackage[spanish]{babel}

% Código fuente embebido
\usepackage{listings}
\lstset{
  basicstyle=\footnotesize,
  showspaces=false,
  showstringspaces=false,
  breaklines=true,
  frame=single
}

% Matemáticos
\usepackage{amssymb,amsmath}

% Tablas complejas
\usepackage{multirow}

% Formato de párrafo
\setlength{\parskip}{1ex plus 0.5ex minus 0.2ex}

% Formato de encabezados y pies de página
\usepackage{fancyhdr}
\fancyhead{}
\fancyfoot{}
\setlength\headheight{40pt}
\fancyhead[L]{
  Facultad de Ingeniería \\ 
  Universidad de Buenos Aires
}
\fancyhead[R]{
  71.14 - Modelos y Optimización I \\
  Trabajo Práctico - 3° entrega
}
\fancyfoot[C]{
  \thepage
}
\pagestyle{fancy}

%%%%%%%%%%%%%%%%%%%%%%%%%%%%%%%%%%%%%%%%%%%%%%%%%%%%%%%%%%%%%%%%%%%%%%%%
% Documento
%%%%%%%%%%%%%%%%%%%%%%%%%%%%%%%%%%%%%%%%%%%%%%%%%%%%%%%%%%%%%%%%%%%%%%%%
\title{\textbf{Trabajo Práctico} - 3° entrega}
\author{Andrés Gastón Arana, \textit{P. 86.203}}
\date{}

\begin{document}

\maketitle
\clearpage

\part{Modelo propuesto por el turno}

\section{Análisis de variaciones de precios y disponibilidades} \label{sec:original}

Se utilizó el software \textit{LINGO} nuevamente para realizar el análisis de
variaciones de precios y disponibilidades de esta sección. A modo de
referencia, se incluye a continuación el modelo utilizado dentro de la
herramienta para dicho análisis:

\lstinputlisting{lingo/original.lng}

De este modelo se obtiene a través del software la solución discutida en
anteriores trabajos prácticos:

\lstinputlisting{lingo/original.out}

\subsection{Curva de oferta y valor del funcional para la línea PF}

De acuerdo a la solución obtenida se observa que la misma es válida para
cualquier coeficiente de \(P_p\) mayor que \(4.91666\), que corresponde a
cualquier precio para la línea PF mayor que \(7.416666\). Para estos valores,
se decide productir y vender \(5000\) kilogramos de producto de dicha línea.

Si variamos el coeficiente hasta \(4.915\) y resolvemos nuevamente el modelo
utilizando el software, obtenemos un nuevo valor de \(P_p\) con su
correspondiente rango de variación. Análogamente se procede resolviendo
versiones modificadas del modelo hasta obtener un coeficiente mínimo, de manera
de acumular los segmentos cubriendo la totalidad del eje real.

A continuación sumarizamos los resultados obtenidos. Cada renglón de la
siguiente tabla implica una corrida del software para detectar los valores de
producción de la línea PF, su correspondiente rango de variación y el valor del
funcional para dicho rango de precios:

\begin{table}[h!]
\centering
\begin{tabular}{ | c | c | c | c | c | c | }
  \hline
  Pmín.                 & Pmáx.     & Cmín.     & Cmáx.     & Producción & Funcional \\ \hline
  -                     & \(7.416\) & -         & \(4.916\) & \(0\)      & \(35500\) \\ \hline
              \(7.416\) & -         & \(4.916\) & -         & \(5000\)   & \(5000 (P - 2.5) + 10916.67\)\\ \hline

  \hline
\end{tabular}
\caption{Variaciones del precio del producto PF}
\end{table}

\FloatBarrier

En esta tabla, P es el precio y C es el coeficiente total resultante en el
funcional (igual a \(P - 2.5\)). La tabla indica que para cualquier precio
menor a \(7.416\) no se produce producto en la línea PF, y el funcional
permanece estable en \(35500\). Para cualquier precio superior a este valor, se
producen \(5000\) unidades de dicho producto, y el funcional varia linealmente
con el precio.

\subsection{Valor marginal para disponibilidad de alfalfa}

Realizando un trabajo análogo al detallado en la sección anterior, se analizará
el valor marginal de la restricción de la cantidad máxima pactada a precio
especial para la alfalfa. Se ubica en la corrida original presentada en la
sección \ref{sec:original} la restricción correspondiente en el modelo
\textbf{PACTADO\_MP\_ALFALFA}.

Cabe destacar que en este caso particular, para la solución óptima no se está
comprando alfalfa para ningún uso. Esto significa que la limitación no tiene
efecto sobre el funcional; no importa si la disponibilidad fuera de \(500kg\),
\(5000kg\) o \(50000kg\), es más económico como fue determinado en entregas
anteriores comprar maiz hasta cubrir los requerimientos protéicos del PF y
completar el kilaje a producir con cáscara de arroz, incluso cuando eso
significa pagar en exceso por el maiz. Esto tiene sentido: el costo de la
alfalfa a precio especial pactado es de \(\$10\) el kilogramo, que es superior
incluso a comprar el maiz con recargo a \(\$5.5\).

Remitiéndonos exclusivamente a la solución calculada por el software, el slack
de la restricción está en cero, y tiene valor marginal \(0\). Más aún, el rango
de valores en los que la base no cambia es de \(0\) a infinito, por lo que este
valor marginal se mantiene para todo valor de la restricción.

\section{Análisis de propuesta económica}

Primeramente, ubicamos el valor de la variable que representa el trabajo en
horas extra en la resolución original presentada en la sección
\ref{sec:original}. Dicha variable toma el valor \(0\), por lo que no se están
usando horas extra. Más aún, la variable DH es de \(1000\), lo que significa
que se están trabajando en total \(13000\) kilogramos, \(1000\) kilogramos por
debajo de los \(14000\) kilogramos mensuales que se pueden fabricar en régimen
regular. El recurso de horas extra, representado por la restricción
\textbf{LIMITE\_HORAS\_EXTRA}, no tiene valor marginal por no estar saturado.

De aquí podemos concluir que, si suponemos que las horas extra no pueden
sustituir las horas de producción en régimen regular, no conviene para ningún
precio pagar por horas extra. El único caso hipotético en el que convendría
adquirir capacidad productiva, que no voy a analizar por considerarlo un caso
degenerado del enunciado, es que el valor hora de lo que ofrece la consultora
tenga un costo menor que el costo de la hora de régimen regular de la empresa.
Sin embargo, este caso no podrían ser consideradas como horas extra y debería
adaptar el modelo para permitir este tipo de decisiones.

\part{Modelo propuesto por el grupo de trabajos prácticos}

\section{Análisis de nuevo producto}

Para introducir el nuevo producto en el problema transcribo primaramente las
tablas simplex iniciales y óptimas del problema:

\begin{table}[h!]
\centering
\begin{tabular}{ | c | c | c || c | c | c | c | c | c | c | }
  \hline
  \(C_i\) & \(X_i\) & \(B_i\) & \(A_1\) & \(A_2\) & \(A_3\) & \(A_4\) & \(A_5\) & \(A_6\) & \(A_m\) \\ \hline
    \(0\) & \(X_3\) & \(15\)  & \(2\)   & \(1\)   & \(1\)   & \(0\)   & \(0\)   & \(0\)   & \(0\) \\ \hline
 \(0\)    & \(X_4\) & \(60\)  & \(6\)   & \(4\)   & \(0\)   & \(1\)   & \(0\)   & \(0\)   & \(0\) \\ \hline
  \(0\)   & \(x_5\) & \(40\)  & \(1\)   & \(3\)   & \(0\)   & \(0\)   & \(1\)   & \(0\)   & \(0\) \\ \hline
  \(-M\)  & \(m\)   & \(12\)  & \(1\)   & \(1\)   & \(0\)   & \(0\)   & \(0\)   & \(-1\)  & \(1\) \\ \hline
\end{tabular}
\caption{Tabla simplex inicial para el problema}
\end{table}

\begin{table}[h!]
\centering
\begin{tabular}{ | c | c | c || c | c | c | c | c | c | }
  \hline
  \(C_i\)  & \(X_i\) & \(B_i\) & \(A_1\) & \(A_2\) & \(A_3\)  & \(A_4\) & \(A_5\)  & \(A_6\) \\ \hline
 \(5000\)  & \(X_1\) & \(1\)   & \(1\)   & \(0\)   & \(0.6\)  & \(0\)   & \(-0.2\) & \(0\) \\ \hline
 \(0\)     & \(X_4\) & \(2\)   & \(0\)   & \(0\)   & \(-2.8\) & \(1\)   & \(-0.4\) & \(0\) \\ \hline
  \(0\)    & \(x_6\) & \(2\)   & \(0\)   & \(0\)   & \(0.4\)  & \(0\)   & \(0.2\)  & \(1\) \\ \hline
  \(3000\) & \(x_2\) & \(13\)  & \(0\)   & \(1\)   & \(-0.2\) & \(0\)   & \(0.4\)  & \(0\) \\ \hline
\end{tabular}
\caption{Tabla simplex óptima para el problema}
\end{table}

\FloatBarrier

En la tabla óptima se removió la columna de \(A_m\), siguiendo la práctica
común de eliminar la columna una vez que salió de la base para evitar cálculos
adicionales que se reflejan en la columna de \(A_6\).

Para obtener la columna de cambio de base, se extraen de la tabla óptima las
columnas que originalmente estaban en la base, tomando en cuenta que la columna
de \(A_6\) tenía el signo cambiado en la identidad original. Luego se
premultiplica el vector de recursos utilizados por esta matriz, para obtener la
nueva columna de la tabla óptima:

\[
  \begin{pmatrix}
    0.6 & 0 & -0.2 & 0 \\
   -2.8 & 1 & -0.4 & 0 \\
    0.4 & 0 & 0.2  & -1 \\
   -0.2 & 0 & 0.4  & 0
  \end{pmatrix}
  *
  \begin{pmatrix}
    4 \\
    10 \\
    2 \\
    1
  \end{pmatrix}
  =
  \begin{pmatrix}
    2 \\
    -2 \\
    1 \\
    0
  \end{pmatrix}
\]

Cuando se incluye esta nueva columna en la tabla óptima, se obtiene una
solución alternativa. Se detallan a continuación las dos tablas óptimas de
ambas soluciones alternativas: 

\begin{table}[h!]
\centering
\begin{tabular}{ | c | c | c || c | c | c | c | c | c | c | }
  \hline
  \(C_i\)  & \(X_i\) & \(B_i\) & \(A_1\) & \(A_2\) & \(A_3\)  & \(A_4\) & \(A_5\)  & \(A_6\) & \(A_7\) \\ \hline
 \(5000\)  & \(X_1\) & \(1\)   & \(1\)   & \(0\)   & \(0.6\)  & \(0\)   & \(-0.2\) & \(0\)   & \(2\) \\ \hline
 \(0\)     & \(X_4\) & \(2\)   & \(0\)   & \(0\)   & \(-2.8\) & \(1\)   & \(-0.4\) & \(0\)   & \(-2\) \\ \hline
  \(0\)    & \(x_6\) & \(2\)   & \(0\)   & \(0\)   & \(0.4\)  & \(0\)   & \(0.2\)  & \(1\)   & \(1\) \\ \hline
  \(3000\) & \(x_2\) & \(13\)  & \(0\)   & \(1\)   & \(-0.2\) & \(0\)   & \(0.4\)  & \(0\)   & \(0\) \\ \hline \hline
  \multicolumn{3}{|c|}{ }      & \(0\)   & \(0\)   & \(2400\) & \(0\)   & \(200\)  & \(0\)   & \(0*\) \\ \hline
\end{tabular}
\caption{Tabla simplex óptima para el problema con el nuevo producto}
\end{table}

\begin{table}[h!]
\centering
\begin{tabular}{ | c | c | c || c | c | c | c | c | c | c | }
  \hline
  \(C_i\)  & \(X_i\) & \(B_i\) & \(A_1\) & \(A_2\) & \(A_3\)  & \(A_4\) & \(A_5\)  & \(A_6\) & \(A_7\) \\ \hline
 \(10000\) & \(X_7\) & \(0.5\) & \(0.5\) & \(0\)   & \(0.3\)  & \(0\)   & \(-0.1\) & \(0\)   & \(1\) \\ \hline
 \(0\)     & \(X_4\) & \(3\)   & \(1\)   & \(0\)   & \(-2.2\) & \(1\)   & \(-0.6\) & \(0\)   & \(0\) \\ \hline
  \(0\)    & \(x_6\) & \(1.5\) & \(-0.5\)& \(0\)   & \(0.1\)  & \(0\)   & \(0.3\)  & \(1\)   & \(0\) \\ \hline
  \(3000\) & \(x_2\) & \(13\)  & \(0\)   & \(1\)   & \(-0.2\) & \(0\)   & \(0.4\)  & \(0\)   & \(0\) \\ \hline \hline
  \multicolumn{3}{|c|}{ }      & \(0*\)  & \(0\)   & \(2400\) & \(0\)   & \(200\)  & \(0\)   & \(0\) \\ \hline
\end{tabular}
\caption{Tabla simplex óptima para el problema con el nuevo producto, solución alternativa}
\end{table}

\FloatBarrier

A primer vista, ambas alternativas son válidas. Sin embargo, la segunda
alternativa implica instalar \(0.5\) máquinas virtuales C, lo que no tiene
sentido. Por lo que en realidad, la única alternativa posible sigue siendo la
solución original, con 1 máquina A y 13 máquinas B.

\part{Heurísticas}

\section{Anális de la situación}

Se trata de un problema de cobertura de conjuntos con particiones, donde las
particiones son las clasificaciones arbitrarias dadas (hombre, mujer, docente,
alumno, etc.) y se quiere minimizar los elementos a utilizar para cubrir el
conjunto completo. Además, existe la dificultad adicional de que se desea
cubrir todo el conjunto de las particiones pero existen limitaciones en cuanto
a qué elementos se pueden elegir en distintas combinaciones, productos del
requerimiento de que todos puedan juntarse un día a la semana.

\section{Objetivo}

Determinar las personas que integrarán el comité en un periodo de tiempo
indeterminado de manera de minimizar la cantidad de personas involucradas.

\section{Hipótesis}

\begin{enumerate}

  \item Todas las personas dentro de cada categoría son equivalentes. El único
    punto a tener en cuenta es seleccionar al menos una persona de cada tipo

  \item Si bien se busca minimizar la cantidad de personas, no hay un límite
    fijo a la cantidad de personas a seleccionar de la lista.

\end{enumerate}

\section{Heurística de resolución}

La heurística desarrollada primeramente determinará el día en el que se
reunirán las personas, de manera de descartar la menor cantidad de personas en
este proceso. Cabe a destacar que este es un proceso a priori, sin considerar
los tipos de persona a seleccionar. Se descartan aquellas personas que no
puedan juntarse en el día seleccionado.

Posteriormente, se toma aquella persona que esté en la mayor cantidad de grupos
no cubiertos. Se repite el proceso hasta que se hayan cubierto todos los
grupos.

El algoritmo heurístico se detalla a continuación:

\begin{enumerate}

  \item Confeccionar una tabla con cada día de la semana (de lunes a domingo),
    y para cada día el número de personas que no pueden reunirse ese día.

  \item Eliminar de la tabla aquellos días para los cuales para alguno de los
    tipos de persona ninguna de las personas con dicho tipo puede reunirse en
    ese día.

  \item Ordenar la tabla por número de personas que no pueden reunirse ese día,
    de menor a mayor. A igual número de personas, ordenar alfabéticamente por
    nombre del día.

  \item Seleccionar el día correspondiente al primer elemento de la tabla
    ordenada en 3. Si la tabla es vacía, el problema es incompatible.

  \item Eliminar de la lista de personas aquellas que no pueden reunirse el día
    seleccionado.

  \item Para cada persona, calcular la cantidad de tipos a los que pertenece
    que están en la lista de tipos a cubrir.

  \item Ordenar la lista de personas de acuerdo al factor calculado en 6, de
    mayor a menor. A igualdad de factores, ordenar alfabéticamente.

  \item Seleccionar la primer persona de la lista ordenada. Eliminarla de la
    lista de personas. Eliminar de la lista de tipos a cubrir los tipos a los
    que pertenece la persona seleccionada.

  \item Si la lista de tipos a cubrir está vacia, el problema está resuelto. En
    caso contrario, volver a 6.

\end{enumerate}

Para el problema particular dado, la solución alcanzada es seleccionar a B, H y E.

\section{Análisis de heurística propuesta}

La heurística es simple y concisa, pero tiene algunas limitaciones serias. En
particular, supone que seleccionar un día para el cual existen la mayor
cantidad de personas disponibles para reunirse permite llegar a una mejor
solución. Es fácil pensar en un ejemplo en el que la heurística diste tanto
como se quiera de la solución óptima. Por ejemplo, si para un día en particular
para el que se pueden reunir pocas personas hay una persona que pertenece a
varios tipos diferentes (por ejemplo, un profesor y simultaneamente alumno que
sólo puede reunirse los viernes), la heurística descartará ese día antes de
realizar ningún tipo de análisis que la llevaría a tomar a esta persona
valuable en el comité.

La otra limitación es que es una heurística golosa que toma inicialmente
aquellas personas que permiten cubrir diferentes tipos, sin tener en cuenta a
futuro que quizás algunas personas son más convenientes porque hay menos
personas que cubran estos roles. A medida que avanza la ejecución, se empiezan
a tomar personas que cubren cada vez menos tipos no cubiertos aún.

\end{document}
