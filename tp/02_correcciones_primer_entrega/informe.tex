\documentclass[a4paper,11pt]{article}

%%%%%%%%%%%%%%%%%%%%%%%%%%%%%%%%%%%%%%%%%%%%%%%%%%%%%%%%%%%%%%%%%%%%%%%%
% Paquetes utilizados
%%%%%%%%%%%%%%%%%%%%%%%%%%%%%%%%%%%%%%%%%%%%%%%%%%%%%%%%%%%%%%%%%%%%%%%%

% Gráficos complejos
\usepackage{graphicx}
\usepackage{caption}
\usepackage{subcaption}
\usepackage{placeins}

% Soporte para el lenguaje español
\usepackage{textcomp}
\usepackage[utf8]{inputenc}
\usepackage[T1]{fontenc}
\DeclareUnicodeCharacter{B0}{\textdegree}
\usepackage[spanish]{babel}

% Matemáticos
\usepackage{amssymb,amsmath}

% Tablas complejas
\usepackage{multirow}

% Formato de párrafo
\setlength{\parskip}{1ex plus 0.5ex minus 0.2ex}

% Formato de encabezados y pies de página
\usepackage{fancyhdr}
\fancyhead{}
\fancyfoot{}
\setlength\headheight{40pt}
\fancyhead[L]{
  Facultad de Ingeniería \\ 
  Universidad de Buenos Aires
}
\fancyhead[R]{
  71.14 - Modelos y Optimización I \\
  Trabajo Práctico - Correcciones 1° entrega
}
\fancyfoot[C]{
  \thepage
}
\pagestyle{fancy}

%%%%%%%%%%%%%%%%%%%%%%%%%%%%%%%%%%%%%%%%%%%%%%%%%%%%%%%%%%%%%%%%%%%%%%%%
% Documento
%%%%%%%%%%%%%%%%%%%%%%%%%%%%%%%%%%%%%%%%%%%%%%%%%%%%%%%%%%%%%%%%%%%%%%%%
\title{\textbf{Trabajo Práctico} - Correcciones 1° entrega}
\author{Andrés Gastón Arana, \textit{P. 86.203}}
\date{}

\begin{document}

\maketitle
\clearpage

\part{Modelo propuesto por el grupo de trabajos prácticos}

\section{Análisis}

El problema propuesto guarda similitud con otros problemas de distribución de
materias primas y determinación de producción óptima para la maximización de
los beneficios por su venta. La dificultad principal del mismo consiste en el
modelado de las condiciones de composición de cada producto de acuerdo a los
nutrientes aportados por cada materia prima, con la particularidad de que la
composición de nutrientes en cada unidad de materia prima es indivisible.

\section{Modificaciones al objetivo}

Dado que la proporción de cada producto no es implícita de acuerdo al objetivo
enunciado en la resolución original del trabajo práctico, se modifica el
objetivo del mismo. El nuevo enunciado es el siguiente:

Determinar la cantidad a producir de cada uno de los productos, las compras de
materias primas correspondientes y la proporción de componentes en cada
producto en un mes para maximizar el beneficio neto obtenido de las ventas de
los productos menos los costos asociados a su producción.

\section{Hipótesis}

Se decidió agregar hipótesis adicionales de acuerdo a los comentarios
observados en la devolución del trabajo práctico.

\begin{enumerate}

  \item No hay obligación alguna para que todos los componentes sean destinados
    a todos los productos. Se posibilita que un producto no contenga ninguna
    cantidad de alguno de los componentes, siempre y cuando cumpla con los
    requisitos de nutrientes pedidos.

   \item Los productos cuyas restricciones a la composición en nutrientes es un
     máximo (como por ejemplo, el máximo de \(40\%\) para los carbohidratos en
     el producto PF) implican que puede darse el caso de que ese producto no
     contenga cantidad alguna de ese nutriente. Sólo se toma como límite
     superior el máximo dado, sin existir límite inferior a dicha proporción.

\end{enumerate}

\section{Modelo}

\subsection{Planteo matemático}

De acuerdo a los comentarios respecto de la presentación del funcional, se
expanden las variables en preparación para la segunda entrega. El funcional
completo es el siguientes:

\[
  \text{MAX} z = 12.5 P_p + 9.5 P_c + 7.5 P_g - 5 C_m - 8 C_a - 10 C_s - 2 C_c - 0.5 E_m - 0.8 E_a - 1 E_s - 0.2 E_c - 1 E_h
\]

En este se han expandido las variables en el funcional de la primer entrega, y
se opero algebraicamente para que cada variable aparezca una única vez con un
único coeficiente.

\part{Modelo propuesto por el turno}

\section{Análisis}

El problema es un problema simple de optimización lineal, modelado con dos
variables únicamente. Es interesante el dominio del problema por estar
relacionado con la informática, pero el modelado de todas las restricciones se
puede reducir a la relación de dos únicas variables, con coeficientes
relativamente pequeños.

\end{document}
